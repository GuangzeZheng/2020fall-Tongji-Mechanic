感谢于颖老师对本小组的精心指导和跟进,老师的建议和督促对本报告的完成提供了极大的帮助。
%通过此次课程设计,我对上学期所学的控制工程基础课程内容有了更进
%一步的理解。通过系统建模、控制器设计、建模仿真等步骤,我对控制工程基
%础所学的知识有了初成体系化的理解,同时也对系统的一些性能参数有了更
%为形象的认识。
%%借着此次课设的机会,我还学到了上学期课内没有学到的知识,即现代控
%制理论方面的基础知识。通过建立系统的状态空间方程以及𝐿𝑄𝑅控制器的设计,
%我对多输入多输出系统有了初步的认识,也算是颇有收获。
%贯穿整个课程设计的,是编程软件𝑀𝐴𝑇𝐿𝐴𝐵的使用,起初由于我对
%𝑀𝐴𝑇𝐿𝐴𝐵不太熟悉,编程起来略有难度,但随着学习的深入,逐渐可以利用软
%件解决一些问题,还有𝑆𝑖𝑚𝑢𝑙𝑖𝑛𝑘工具箱的使用,这是一个图形化编程软件,功
%能强大,掌握好它的使用必将对后续的学习有极大的帮助。




%衷心感谢导师 xxx 教授和物理系 xxx 副教授对本人的精心指导。他们的言传身教将使
%我终生受益。

%在美国麻省理工学院化学系进行九个月的合作研究期间,承蒙 xxx 教授热心指导与帮助,不
%胜感激。感谢 xx 实验室主任 xx 教授,以及实验室全体老师和同学们的热情帮助和支
%持!本课题承蒙国家自然科学基金资助,特此致谢。

%感谢 \tongjithesis{},它的存在让我的论文写作轻松自在了许多,让我的论文格式规整漂亮了
%许多。