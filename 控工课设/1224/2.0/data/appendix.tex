\chapter{LQR在matlab中的代码}
\label{cha:engorg}
\begin{lstlisting}
A=[0 1 0 0; 0 0 0 0; 0 0 0 1; 0 0 29.4 0];
B=[0 1 0 3]';
C=[1 0 0 0; 0 0 0 1];
D=[0 0]';
Co=ctrb(A,B);
rank(Co)
Ob=obsv(A,C);
rank(Ob)
\end{lstlisting}

\begin{lstlisting}
A=[0 1 0 0;0 0 0 0;0 0 0 1;0 0 29.4 0];
B=[0 1 0 3]';
C=[1 0 0 0;0 0 1 0];
D=[0 0 ]';
Q11=1000;
Q33=100;
Q=[Q11 0 0 0;0 0 0 0;0 0 Q33 0;0 0 0 0];
R=1;
K=lqr(A,B,Q,R);
Ac=(A-B*K);
T=0:0.001:5;
U=0.2*ones(size(T));
[Y,X]=lsim(Ac,B,C,D,U,T);
plot(T,X(:,1),'-g','LineWidth',1);
hold on;
plot(T,X(:,2),'-.b','LineWidth',1);
plot(T,X(:,3),'-','LineWidth',1);
plot(T,X(:,4),'-r','LineWidth',1);
hold off;
legend('x','dotx','fai','dotfai');
\end{lstlisting}