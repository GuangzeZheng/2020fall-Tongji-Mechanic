\chapter{前期准备与相关工作}
\section{前期准备}

倒立摆系统的控制,可以采用经典的PID控制方法,LQR控制,模糊控制法,神经网络控制算法,根轨迹控制算法等,我们组搜集了LQR,模糊算法,神经网络算法,以及PID相关方面的资料,进行比较整理,从四种方法中选取了LQR和模糊控制算法进行研究设计。并对研究内容,进度规划,报告,答辩进行了细致的分工,按照计划甘特图推进课程设计,如图~\ref{fig:gant}所示。

\begin{figure}[h]
\centering
    \includegraphics[width=12cm]{gant.pdf}
    \caption{任务计划甘特图}
    \label{fig:gant}
\end{figure}


根据查找的文献资料,我们简单总结了四种控制方法的优缺点,现罗列如下

\subsection{LQR(线性二次型调节器)}
 LQR具有较好的控制住摆杆,并且响应速度较快,具有超调量。该方法可以使目标函数达到最优,可以对能控系统进行任意的极点配置来满足所设计系统的性能要求,提高闭环系统的相对稳定性或者使不稳定系统得以镇定,同时具有较强的鲁棒性$^{[1]}$。但对小车的控制效果稍差,并且LQR需要调整两个矩阵,要求解$Riccati$方程确定$Q$和$R$权矩阵,算法复杂$^{[2]}$。

\subsection{模糊控制}
基于模糊控制的方法使用语言方法,可不需要过程的精确数学模型;鲁棒性强,适于解决过程控制中的非线性、强耦合时变、滞后等问题;
有较强的容错能力。具有适应受控对象动力学特征变化、环境特征变化和动行条件变化的能力。但是模糊控制的设计尚缺乏系统性,这对复杂系统的控制是难以奏效的。难以建立一套系统的模糊控制理论,以解决模糊控制的机理、稳定性分析、系统化设计方法等一系列问题;如何获得模糊规则及隶属函数即系统的设计办法,完全凭经验进行;信息简单的模糊处理将导致系统的控制精度降低和动态品质变差。若要提高精度就必然增加量化级数,导致规则搜索范围扩大,降低决策速度,甚至不能进行实时控制$^{[3]}$。

\subsection{神经网络}
基于神经网络的算法属于非线性映射,能以任意精度逼近任何非线性连续函数,适合求解内部机制复杂问题。
而且输入输出变量数目是任意的,具有自学习和自适应的能力,能过学习获取输出数据间的对应关系,将学习内容存储到网络权值中,具有容错能力,部分神经元受损对全局训练结果不会有很大影响。但是该方法存在实时性和自适应性相互矛盾的问题,不能保证快速性和有效性;并且权值容易收敛到局部最小点,收敛速度慢,隐含层数目难以确定,训练依赖样本数据,样本数据有采集难度$^{[4]}$。

\subsection{PID控制}
基于PID原理的控制系统结构简单,易于实现,使用方便,PID各参数相互独立,可以根据过程的动态特性及时调节;
适用性强,可通过适当简化将非线性的、时变的被控对象变成基本线性和动态特性不随时间变化的系统,应用范围十分广泛,理论成熟。
棒性较好,即其控制品质对被控对象特性的变化不太敏感$^{[5]}$。
但是该方法稳定性差,在控制非线性、时变、耦合及参数和结构不确定的复杂过程时,效果不好$^{[6]}$。

\section{题目数据}

将题目的数据整理如表~\ref{para}所示。

\begin{table}[h]
\centering
\begin{tabular}{ccc}
\hline
参数 & 意义           & 数值                            \\ \hline
$M$  & 小车质量         & $1.096kg$                       \\ \hline
$m$ & 摆杆质量         & $0.109kg$                       \\ \hline
$l$  & 摆杆质心到转动轴心的长度 & $0.25m$                         \\ \hline
$b$  & 摩擦比例系数       & $0.1N.s/m$                      \\ \hline
$I$  & 摆杆对质心的转动惯量   & $0.0034kg.m^2$ \\ \hline
$T$  & 采样时间         & $0.005s$                        \\ \hline
\end{tabular}
\caption{题目参数}\label{para}
\end{table}

\chapter{物理模型的建立和状态空间公式的推导}

\section{模型假设与分析}
\subsection{受力分析}
为了建立物理模型,现有如下假设:
1、摆杆质量均匀,质心位于其几何中心处
2、忽略除b以外的所有摩擦力

如图\ref{fig:car}对小车进行受力分析,以摆杆和小车交点为原点,以水平向右和竖直向下为正方向建立坐标系,沿$x$轴有牛顿第二定律

\begin{figure}[hbpt]
\centering
\includegraphics[width=12cm]{car.jpg}
\caption{小车受力分析}\label{fig:car}
\end{figure}


\begin{equation}
F-b\dot x-N_x^{'}=M\ddot x
\end{equation}

如图\ref{fig:stick}对摆杆进行受力分析,沿$x,y$轴有牛顿第二定律和沿$\theta$向动量矩定理

\begin{figure}[hbpt]
\centering
\includegraphics[width=12cm]{stick.jpg}
\caption{摆杆受力分析}\label{fig:stick}
\end{figure}

摆杆质心$c$点
	
\begin{equation}
\begin{aligned}
x_c&=x+lsin\theta\\
y_c&=lcos\theta\\
\end{aligned}
\end{equation}

\subsection{加速度分析}
求导可得质心加速度

\begin{equation}
\begin{aligned}
\ddot x_c&=\ddot x+l\ddot{\theta}cos\theta-l\dot{\theta}^2sin\theta\\
\ddot y_c&=-l\ddot{\theta}sin\theta-l\dot{\theta}^2cos\theta\\
\end{aligned}
\end{equation}
\begin{equation}
\begin{aligned}
&N_y-mg=-m\ddot y_c\\
&N_x=m\ddot x_c\\
&N_ylsin\varphi+N_xlcos\varphi=I\ddot{\varphi}\\
\end{aligned}
\end{equation}

应用小量近似$sinx\doteq x$,$cosx=1-\frac{x^2}{2}$以及$\theta=\varphi+\pi$这个关系整理以上各式,可得倒立摆系统物理方程组

\begin{equation}
\begin{aligned}
&F=(M+m)\ddot x-ml\ddot{\varphi}+b\dot x\\
&(I+ml^2)\ddot{\varphi}=ml\ddot x+mgl\varphi\\
\end{aligned}
\end{equation}

\subsection{传递函数求解}
拉氏变换可得

\begin{equation}
\begin{aligned}
&(M+m)X(s)s^2+bX(s)s-ml\varphi(s)s^2=F(s)\\
&(I+ml^2)\varphi(s)s^2-mgl\varphi((s)=mlX(s)s^2\\
\end{aligned}
\end{equation}

注意到小车加速度$A(s)=X(s)s^2$

可得从小车角速度输入到摆杆角度输出的传递函数

\begin{equation}
\frac{\varphi(s)}{A(s)}=\frac{ml}{(I+ml^2)s^2-mgl}
\end{equation}

进一步整理,可以得到力输入到摆杆角度和小车位移的传递函数

\begin{equation}
\begin{aligned}
&\frac{\varphi(s)}{F(s)}=\frac{mls^2}{[(I+ml^2)(M+m)-m^2l^2]s^4+b(I+ml^2)s^3-(M+m)mgls^2-bmgls}\\
&\frac{X(s)}{F(s)}=\frac{(I+ml^2)s^2-mgl}{[(I+ml^2)(M+m)-m^2l^2]s^4+b(I+ml^2)s^3-(M+m)mgls^2-bmgls}
\end{aligned}
\end{equation}

\section{状态空间求解}

将物理方程组进行等价变形,可以得到

\begin{equation}
\begin{aligned}
&\ddot x=-\frac{bI+bml^2}{\Delta}\dot x+\frac{m^2gl^2}{\Delta}\varphi+\frac{I+ml^2}{\Delta}F\\
&\ddot{\varphi}=\frac{-mlb}{\Delta}\dot x+\frac{mg(M+m)l}{\Delta}\varphi+\frac{ml}{\Delta}F\\
\end{aligned}
\end{equation}

其中,$\Delta=I(M+m)+Mml^2$.

基于此,取$z_1=x,z_2=\dot x,z_3=\varphi,z_4=\dot{\varphi}$为状态空间变量,以力$F$作为输入$u$,建立状态空间矩阵

\begin{equation}
\begin{aligned}
&\begin{bmatrix}
\dot x\\
\ddot x\\
\dot{\varphi}\\
\ddot{\varphi}\\
\end{bmatrix}
=
\begin{bmatrix}
0 & 1 & 0 & 0\\
0 & -\frac{bI+bml^2}{\Delta} & \frac{m^2gl^2}{\Delta} & 0\\
0 & 0 & 0 & 1\\
0 & \frac{-mlb}{\Delta} & \frac{mg(M+m)l}{\Delta} & 0\\
\end{bmatrix}
\begin{bmatrix}
x\\
\dot x\\
\varphi\\
\dot{\varphi}\\
\end{bmatrix}
+
\begin{bmatrix}
0\\
\frac{I+ml^2}{\Delta}\\
0\\
\frac{ml}{\Delta}\\
\end{bmatrix}
u\\
&\begin{bmatrix}
x\\
\varphi\\
\end{bmatrix}
=
\begin{bmatrix}
1 &0 &0 &0\\
0 &0 &1 &0\\
\end{bmatrix}
\begin{bmatrix}
x\\
\dot x\\
\varphi\\
\dot{\varphi}\\
\end{bmatrix}
+
\begin{bmatrix}
0\\
0\\
\end{bmatrix}
u\\
\end{aligned}
\end{equation}

注意到该状态空间矩阵较为复杂,若取小车加速度作为输入$u^{'}$,可以简化该状态空间矩阵

根据转动惯量的定义式,并认为摆件质地均匀,有下式

\begin{equation}
I=\frac{1}{12}m(2l)^2
\end{equation}

带入整理,得

\begin{equation}
\ddot \varphi=\frac{3g}{4l}\phi+\frac{3}{4l}\ddot x
\end{equation}

则可得较为简单的状态空间矩阵

\begin{equation}
\begin{aligned}
&\begin{bmatrix}
\dot x\\
\ddot x\\
\dot{\varphi}\\
\ddot{\varphi}\\
\end{bmatrix}
=
\begin{bmatrix}
0 & 1 & 0 & 0\\
0 & 0 & 0 & 0\\
0 & 0 & 0 & 1\\
0 & 0 & \frac{3g}{4l} & 0\\
\end{bmatrix}
\begin{bmatrix}
x\\
\dot x\\
\varphi\\
\dot{\varphi}\\
\end{bmatrix}
+
\begin{bmatrix}
0\\
1\\
0\\
\frac{3}{4l}\\
\end{bmatrix}
u'\\
&\begin{bmatrix}
x\\
\varphi\\
\end{bmatrix}
=
\begin{bmatrix}
1 &0 &0 &0\\
0 &0 &1 &0\\
\end{bmatrix}
\begin{bmatrix}
x\\
\dot x\\
\varphi\\
\dot{\varphi}\\
\end{bmatrix}
+
\begin{bmatrix}
0\\
0\\
\end{bmatrix}
u^{'}\\
\end{aligned}
\end{equation}

代入数据可得

\begin{equation}
\begin{aligned}
&\begin{bmatrix}
\dot x\\
\ddot x\\
\dot{\varphi}\\
\ddot{\varphi}\\
\end{bmatrix}
=
\begin{bmatrix}
0 & 1 & 0 & 0\\
0 & 0 & 0 & 0\\
0 & 0 & 0 & 1\\
0 & 0 & 29.4 & 0\\
\end{bmatrix}
\begin{bmatrix}
x\\
\dot x\\
\varphi\\
\dot{\varphi}\\
\end{bmatrix}
+
\begin{bmatrix}
0\\
1\\
0\\
3\\
\end{bmatrix}
u'\\
&\begin{bmatrix}
x\\
\varphi\\
\end{bmatrix}
=
\begin{bmatrix}
1 &0 &0 &0\\
0 &0 &1 &0\\
\end{bmatrix}
\begin{bmatrix}
x\\
\dot x\\
\varphi\\
\dot{\varphi}\\
\end{bmatrix}
+
\begin{bmatrix}
0\\
0\\
\end{bmatrix}
u^{'}\\
\end{aligned}
\end{equation}

\chapter{用PID算法校正直线一级倒立摆系统}
%-------------------------------------------------------------------
\section{PID控制分析}

\subsection{PID介绍}

PID控制就是用线性组合的方式,把偏差的比例$P$、积分$I$、微分$D$组合构成控制量。对被控对象展开控制的方法。在PID控制器中,通过比例单元$P$将偏差进行比例放大得到输出,但通过这一过程无法消除余差,因此加以积分单元$I$,积分依照偏差累计,只要当偏差不为0时,积分值就不为0,考虑到偏差变化有速度快慢之分,加以微分单元$D$,计算偏差变化的速率,PID控制就是综合使用这三个单元来控制被控变量。其原理控制示意图如图~\ref{fig:PID_principle}所示。

\subsection{PID控制器原理性推导}
	
PID控制器是一种线性控制器,其根据给定值$r(t)$与实际输出值$y(t)$构成的控制偏差$e(t)$为:

\begin{figure}[hbpt]
\centering
\includegraphics[width=12cm]{PID_principle.png}
\caption{PID原理图}\label{fig:PID_principle}
\end{figure}

\begin{equation}
e(t)=r(t)-y(t)
\end{equation}

其输入控制偏差$e(t)$与输出控制结果$u(t)$的关系为:

\begin{equation}
u(t)=K_pe(t)+K_I\int_0^te(t)dt+K_D \frac{de(t)}{dt}
\end{equation}

上式进行拉氏变换,得其传递函数为:

\begin{equation}
\begin{aligned}
G(s)&=\frac{U(s)}{E(s)}\\
&=K_p+\frac{1}{K_Is}+K_Ds\\
&=\frac{K(\tau_1s+1)(\tau_2s+1)}{s}
\end{aligned}
\end{equation}

其中,$K_pe(t)$为比例环节,随着$K_p$的增加,可以更好地减小偏差,但同时$K_p$还影响系统的稳定性,$K_p$增加通常导致系统的稳定性下降,过大的$K_p$往往使系统产生剧烈的震荡和不稳定。

$K_I\int_0^te(t)dt$为积分环节,消除系统静态误差,作用的强弱由$K_I$决定,$K_I$越大,积分作用越强,反之则越弱,但同时积分环节也可能增大系统超调量。

$K_D\frac{de(t)}{dt}$为微分环节,针对被测量的变化速率来进行调节,预测偏差信号的变化趋势,在其出现较大变化之前引入修正信号与之低效,从而减小系统的调节时间。

\section{实验分析}

实验室倒立摆控制系统结构图如图~\ref{fig:structure}所示。

\begin{figure}[hbpt]
\centering
\includegraphics[width=12cm]{structure.png}
\caption{实验室倒立摆控制系统结构图}\label{fig:structure}
\end{figure}

修改PID各项参数,通过角度编码器测量摆杆的摆动角度,通过伺服电机控制小车的位移速度和加速度,通过控制器利用摆杆的惯性力控制摆杆的位移速度和加速度,从而控制摆杆的角度,最终可以实现直线倒立摆的竖直稳定.

当其受到外界干扰时,在干扰停止作用后,系统能够很快地回到平衡位置。但是,整个控制系统中并无小车位移的反馈,只能通过角度编码器获取摆杆的角度,通过传动比转换近似得到小车的位移。因此,PID控制器无法对小车的位置偏差进行修正,不能对小车的位置进行控制,当受到扰动时,小车会沿滑轨一直向扰动方向运动,撞到滑轨边缘,无法恢复到初始平衡位置。后续考虑使用其它控制方法,既能实现直线倒立摆的竖直稳定,又可以控制小车位置的稳定不变。

\chapter{LQR线性二次型调节器}


\section{LQR介绍}

LQR (linear quadratic regulator,线性二次型调节器) 利用现代控制理论中以状态空间矩阵形式给出的线性系统,利用目标函数(能量函数)$J=\frac{1}{2}\int_0^\infty(x^TQx+u^TRu)dt$(其中Q为半正定矩阵,R为正定矩阵),设计状态反馈控制器$K$使得目标函数的值最小。
LQR控制器可以在系统偏离平衡状态时,尽可能减少消耗的能量保持系统状态各分量仍接近平衡状态.以一维系统$X=x(t)$为例,则$x^TQx=Qx^2$,为了使得$J$最小,那么该函数一定有界,故有下式

\begin{equation}
\lim_{t \rightarrow \infty}x(t)=0
\end{equation}

这保证了系统的稳定性,类似的$u(t)$小保证了节省能量,控制代价降低。

\section{LQR控制器原理性推导}

线性系统的状态空间可以描述为
	
\begin{equation}
\begin{aligned}
\dot X=AX+Bu\\
Y=CX+Du\\
\end{aligned}
\end{equation}

评价函数为

\begin{equation}
J=\frac{1}{2}\int_0^\infty(x^TQx+u^TRu)dt
\end{equation}

$Q、R$分别是对状态变量和输入量的加权矩阵,确定误差和能量损耗的相对性。

根据极小值原理,引入$n$维协态矢量$\lambda(t)$,构造哈密顿函数$^{[9]}$

\begin{equation}
H(x,u,\lambda)=\frac{1}{2}[x^TQx+u^TRu]+\lambda^T[Ax+Bu]
\end{equation}

最优控制使得$H$取极值,即

\begin{equation}
\frac{\partial H}{\partial u}=Ru+B^T\lambda=0
\end{equation}

解得

\begin{equation}
u=-R^{-1}B^T\lambda
\end{equation}

又有

\begin{equation}
\frac{\partial^2 H}{\partial u^2}=R
\end{equation}

R正定,故上式为系统的最优控制律。

设$\lambda=Px$,$P$为$n$阶实对称正定矩阵,且满足黎卡提矩阵代数方程

\begin{equation}
-PA-A^TP+PBR^{-1}B^T-Q_1=0
\end{equation}

则最优控制

\begin{equation}
u=-R^{-1}B^T\lambda=-Kx
\end{equation}

系统最优轨线为

\begin{equation}
\dot x(t)=(A-BK)x(t)
\end{equation}

在$matlab$中可以利用$lqr$函数求得反馈矩阵$K$.

\section{LQR的系统能控性和能观性分析}

对于上面假设的线性系统,状态完全能控制的充要条件是$^{[7]}$

\begin{equation}
rank[B,AB,A^2B,\dots A^{n-1}B]=n
\end{equation}

系统状态能够完全观测的充要条件是

\begin{equation}
rank
\begin{bmatrix}
C\\
CA\\
CA^2\\
\vdots\\
CA^{n-1}\\
\end{bmatrix}
=n
\end{equation}

\section{小车系统权重的选取以及能控性能观性分析}

前面得到小车的状态空间方程为

\begin{equation}
\begin{aligned}
&\begin{bmatrix}
\dot x\\
\ddot x\\
\dot{\varphi}\\
\ddot{\varphi}\\
\end{bmatrix}
=
\begin{bmatrix}
0 & 1 & 0 & 0\\
0 & 0 & 0 & 0\\
0 & 0 & 0 & 1\\
0 & 0 & 29.4 & 0\\
\end{bmatrix}
\begin{bmatrix}
x\\
\dot x\\
\varphi\\
\dot{\varphi}\\
\end{bmatrix}
+
\begin{bmatrix}
0\\
1\\
0\\
3\\
\end{bmatrix}
u'\\
&\begin{bmatrix}
x\\
\varphi\\
\end{bmatrix}
=
\begin{bmatrix}
1 &0 &0 &0\\
0 &0 &1 &0\\
\end{bmatrix}
\begin{bmatrix}
x\\
\dot x\\
\varphi\\
\dot{\varphi}\\
\end{bmatrix}
+
\begin{bmatrix}
0\\
0\\
\end{bmatrix}
u^{'}\\
\end{aligned}
\end{equation}

在$matlab$中(代码见附录)进行矩阵秩计算可知两个判定矩阵的秩都是$4$,则小车倒立摆系统可控可观测。

\section{小车倒立摆仿真}

首先根据系统的运动微分方程在$matlab$中的$simulink$进行仿真$^{[8]}$,由上述矩阵导出参考方程,为方便,将一些量名称更改如下

\begin{equation}
\begin{aligned}
&\dot x\rightarrow \dot x_1\\
&\ddot x\rightarrow \dot x_2\\
&\dot \varphi \rightarrow \dot x_3\\
&\ddot \varphi \rightarrow \dot x_4\\
\end{aligned}
\end{equation}

则有

\begin{equation}
\begin{aligned}
&\dot x_1=x_2\\
&\dot x_2=u\\
&\dot x_3=x_4\\
&\dot x_4=29.4x_3+3u\\
&u=-k_1x_1-k_2x_2-k_3x_3-k_4x_4\\
\end{aligned}
\end{equation}

得到如\ref{LQR_simulink}的仿真

\begin{figure}[hbpt]
\centering
\includegraphics[width=16cm]{LQR_simulink.png}
\caption{$LQR的simulink$}\label{LQR_simulink}
\end{figure}

这种仿真可以通过给定$x_i$的初始值来模拟脉冲激励

\begin{figure}[hbpt]
\centering
\includegraphics[width=12cm]{initial.png}
\caption{模拟脉冲激励}\label{initial}
\end{figure}

给定小车$5$单位位移,摆杆$5$单位角度,可以得到如\ref{1000_100}的响应曲线。

\begin{figure}[hbpt]
\centering
\includegraphics[width=12cm]{1000_100.png}
\caption{脉冲激励仿真结果}\label{1000_100}
\end{figure}

可以发现系统可以稳定。关于修改权重以及关于曲线的比较分析在后面介绍。

此外也可以利用编程来模拟仿真,以$lsim$函数模拟的阶跃信号为例,编写代码(见附件)也可以得到响应曲线如\ref{1000_100_2}

\begin{figure}[hbpt]
\centering
\includegraphics[width=12cm]{1000_100_2.png}
\caption{阶跃激励仿真结果}\label{1000_100_2}
\end{figure}

\subsection{修改权重分析比较}

依次修改$Q_{11}Q_{22}Q_{33}Q_{44}$权重(值见图片),建立如\ref{compare}的仿真

\begin{figure}[hbpt]
\centering
\includegraphics[width=12cm]{compare.png}
\caption{比较仿真}\label{compare}
\end{figure}

观察比较仿真结果如\ref{data}
\begin{figure}[hbpt]
\centering
\includegraphics[width=12cm]{x.png}
\includegraphics[width=12cm]{dotx.png}
\includegraphics[width=12cm]{fai.png}
\includegraphics[width=12cm]{dotfai.png}
\caption{数据比较}\label{data}
\end{figure}

不难发现这三组数据都可以保证小车和摆杆的稳定性,都可以比较快速的达到稳定状态,区别就是随着权重的增大,控制目标变量可以更快的达到稳定,并且超调量变小,但是权重的变大意味着输入的能耗增加(如\ref{u}),不是最经济的选择.

\begin{figure}[hbpt]
\centering
\includegraphics[width=12cm]{u.png}
\caption{u}\label{u}
\end{figure}

因此,在可以快速的稳定这个前提下,应该尽可能减小权重值。最终我们选定$Q11=1000,Q33=100$这个权重值,其对应$K$矩阵为

\begin{equation}
K=
\begin{bmatrix}
-31.6228\\
-19.8928\\
-71.1181\\
12.9858\\
\end{bmatrix}
\end{equation}

\chapter{模糊控制调节器}

\section{模糊控制介绍}
模糊控制诞生于上世纪七十年代,相比于传统控制理论,模糊控制在面对难以使用精确数学方法描述模型的情况下有更大的优势。模糊控制能够与其他方法结合,如模糊PID控制、自适应模糊控制等,表现出良好相容性。模糊控制以模糊数学为基础,使用模糊集合论、模糊语言以及模糊逻辑 ,将日常经验与相关知识结合到控制器中,形成类似人的思考决策过程$^{[10]}$。

\section{模糊控制原理}
模糊控制系统由输入输出装置、模糊控制器、执行机构、受控对象、监测装置组成,结构与一般控制系统类似。模糊控制器由模糊化、模糊推理、解模糊、知识库组成$^{[3]}$。
\subsection{量化因子}
量化因子等于模糊集论域除以基本论域,其作用是将实际输入量的真实论域转化为模糊论域。
\subsection{比例因子}
类似的,比例因子等于预期输出量除以模糊集输出量,将模糊控制器输出转化为适当大小实际输出。
\subsection{模糊化}
通过隶属度函数,将精确的输入量得到其对对各模糊集的隶属度,为模糊推理提供依据。常用隶属度函数有s形、z形、高斯形、三角形等。
\subsection{知识库}
知识库分为数据库与规则库。
数据库主要包括模糊化、模糊推理、解模糊的一切知识及模糊空间的分级数等。 
规则库则是模糊集的映射关系,由专家知识以及经验等构成,即推理规则。
\subsection{推理机}
根据给定推理规则,结合当前输入与模糊关系给出模糊输出量。推理机是模糊控制器的核心部分。
\subsection{解模糊}
将从推理机处接受到的模糊量解模糊,清晰化为论域内的清洗量,并经过比例因子,变换为实际输出量。解模糊的方法主要有重心法、最大隶属度法等。Matlab提供了五种解模糊方法:面积重心法、面积等分法、最大隶属度平均法、最大隶属度取小法、最大隶属度取大法。此次实验使用了重心解模糊法。

\section{设计}
为了同时控制角度与位置,我们设计了两个模糊控制器,将两个控制器输出和作为最终输出,以角度控制输出为主,以达到控制目的。
按照经验划分,我们将角度误差范围、角速度误差范围、输出范围分别定为
[-0.25,0.25], [-2,2], [-20,20];将位移误差范围、位移误差范围、输出范围分别定为
[-0.4,0.4], [-1.5,1.5], [-20,20]。同时,依据如表~\ref{mohu}建立了控制器。

\begin{table}[h]
    \centering
    \fontsize{8}{10}\selectfont    
    \caption{模糊控制规则表}
\begin{tabular}{l|ccccc}
    \toprule
\diagbox [width=5em,trim=l] {v/$\omega$}{x/$\theta$} & NB & NS & ZE & PS & PB   \\
\hline
NB & NB & NM & NM & NS & ZE   \\
ZE & NM & NS & ZE & PS & PM   \\
PB & ZE & PS & PM & PM & PB   \\
    \bottomrule
\end{tabular}\vspace{0cm}
    \label{mohu}
\end{table}

角度控制器,依次为角度、角速度、输出量隶属度函数,如图~\ref{Fig:a}所示

\begin{figure}[hbpt]
\centering
\includegraphics[width=9cm]{a1.png}
%\caption{角度隶属度函数}
\includegraphics[width=9cm]{a2.png}
%\caption{角速度隶属度函数}
\includegraphics[width=9cm]{a3.png}
%\caption{输出量隶属度函数}
\caption{角度、角速度、输出量隶属度函数}\label{Fig:a}
\end{figure}

\section{仿真结果分析}
\subsection{在无干扰的情况下仅控制角度}
在仅通过反馈角度进行控制的条件下,角度能够保持稳定,稳定在10-3的量级上。而位移则不受控制,持续偏向一边,控制效果不理想,仿真和效果分别如图~\ref{fig:onlyanglesim},图~\ref{fig:onlyangle}所示。
\begin{figure}[hbpt]
\centering
\includegraphics[width=9cm]{onlyangle1.png}
\caption{在无干扰的情况下仅控制角度的simulink仿真}\label{fig:onlyanglesim}
\end{figure}

\begin{figure}[hbpt]
\centering
%\includegraphics[width=9cm]{onlyangle1.png}
\includegraphics[width=9cm]{onlyangle2.png}
\includegraphics[width=9cm]{onlyangle3.png}
\caption{在无干扰的情况下仅控制角度的控制结果}\label{fig:onlyangle}
\end{figure}

\subsection{在有干扰的情况下仅控制角度}
与无干扰类似,角度仍然能够很好的稳定下来,但是小车将持续偏向一边,仿真和效果分别如图~\ref{fig:disturbsim},图~\ref{fig:disturb}所示。

\begin{figure}[hbpt]
\centering
\includegraphics[width=9cm]{dis1.png}

\caption{在有干扰的情况下仅控制角度的控制结果}\label{fig:disturbsim}
\end{figure}


\begin{figure}[hbpt]
\centering

\includegraphics[width=9cm]{dis2.png}
\includegraphics[width=9cm]{dis2.png}
\caption{在有干扰的情况下仅控制角度的控制结果}\label{fig:disturb}
\end{figure}

\subsection{在无干扰的情况下同时控制角度与位移}
在加入对位移的控制后,位移也得到了很好的控制,如图~\ref{fig:x}所示。

\begin{figure}[hbpt]
\centering
\includegraphics[width=9cm]{x1.png}
\includegraphics[width=9cm]{x2.png}
\caption{在无干扰的情况下同时控制角度与位移的控制结果}\label{fig:x}
\end{figure}

\chapter{设计经验与心得体会}
\section{设计经验}
LQR控制器的设计过程中,权重矩阵也是要通过多次尝试,才能找到较
为适宜的值。在模糊控制中,xxxxxxxxxxxxxxxxxxxxxxxxxxx。
此次设计过程中尝试了多种控制方法,这些方法各有特点,在控制系统设
计的时候要灵活选用,方能较好地达到控制目标。
\section{心得体会}
通过此次课程设计,本组成员对本学期所学的控制工程基础课程内容有了更进
一步的理解。通过系统建模、控制器设计、建模仿真等步骤,本组成员对控制工程基
础所学的知识有了初成体系化的理解,同时也对系统的一些性能参数有了更
为形象的认识。借着此次课设的机会,本组成员还学到了上学期课内没有学到的知识,即现代控
制理论方面的基础知识。通过建立系统的状态空间方程以及LQR控制器的设计、模糊控制设计,
本组成员对多输入多输出系统有了初步的认识。编程软件MATLAB的使用贯穿整个课程设计,起初由于本组成员对
MATLAB不太熟悉,编程起来略有难度,但随着学习的深入,逐渐可以利用软
件解决一些问题,还有Simulink工具箱的使用,这是一个图形化编程软件,功
能强大。掌握好这些软件的使用必将对后续的学习有极大的帮助。






